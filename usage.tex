\documentclass{peterlitsdoc}
\newcommand{\vb}{\verb}

\title{\vb|peterlitsdoc|文档类}
\author{Peterlits Zo}

\begin{document}

\maketitle
\tableofcontents

%%%%%%%%%%%%%%%%%%%%%%%%%%%%%%%%%%%%%%%%%%%%%%%%%%%%%%%%%%%
%%%%%%%%%%%%%%%%%%%%%%%%%%%%%%%%%%%%%%%%%%%%%%%%%%%%%%%%%%%

\section{前言}

出于一些原因,我开始使用markdown来写文档,然后使用pandoc来转换为PDF文档。

但是转换成的PDF文档格式有点不尽人意。
后来为了更棒的样式,我开始使用Python写了一个filter来操控中间层json数据。

后来python的filter的数据越来越多,比如我在meta data区中写下了很多我平时会
用到的短命令(比如,添加图片,带标注的段落等等),pandoc会把code block设置为
单独的一个段,我还递归修改让它依附上上一个段,来让它有着合理的前后间距。

那为什么不直接一开始就写\LaTeX{}文档呢?因为我一开始觉得\LaTeX{}的语法好傻哦,
后来又看了看Lisp才慢慢看到\LaTeX{}的美感,于是之后我把我写的filter的程序
中可能会用到的命令和繁琐的设置全部都放到这个文档类中。

这个文档类的目的是写一个面向中文使用的漂亮的小文章。

这个文档类的设计哲学是:命令具体,尽可能覆盖住需要的部分,在它的view层,
需要做到简洁具体。不应该太过于花里胡哨。

欢迎在\url{https://github.com/PeterlitsZo/peterlitsdoc}中提交issue来让我添
加一些有用的功能,或者下载相应的\vb|peterlitsdoc.cls|,或者提交pull request
来改进(欢迎$\sim$)。

基于GPL发布。

%%%%%%%%%%%%%%%%%%%%%%%%%%%%%%%%%%%%%%%%%%%%%%%%%%%%%%%%%%%
%%%%%%%%%%%%%%%%%%%%%%%%%%%%%%%%%%%%%%%%%%%%%%%%%%%%%%%%%%%

\section{peterlitsdoc所提供命令}

%%%%%%%%%%%%%%%%%%%%%%%%%%%%%%%%%%%%%%%%%%%%%%%%%%%%%%%%%%%

\subsection{pltpara命令和它的快捷命令 - 带标注的段落}

使用\vb|\pltpara{name}{conten}|来做一个peterlitsdoc风格的带注释自然段。
\begin{pltrun}
\pltpara{测试}{测试}
\end{pltrun}

还定义了\vb|\pltrit|和\verb|\pltnte|来作为订正和批注命令。
\begin{pltrun}
\pltrit{测试}
\pltnte{测试}
\end{pltrun}

命令是可以叠加的。
\begin{pltrun}
\pltnte{\pltrit{\pltnte{测试}改错}批注}
\end{pltrun}

%%%%%%%%%%%%%%%%%%%%%%%%%%%%%%%%%%%%%%%%%%%%%%%%%%%%%%%%%%%

\subsection{pltendant命令 - 文末的作者信息}

使用类似于\vb|\plttendant{yyyy}{mm}{dd}{name}|的格式来定义结束的名字。
\begin{pltrun}
这就是《瓦尔登湖》所解释的哲学。

and the winner is: la vie.

\pltendant{2020}{5}{25}{peterlits zo}
\end{pltrun}

%%%%%%%%%%%%%%%%%%%%%%%%%%%%%%%%%%%%%%%%%%%%%%%%%%%%%%%%%%%

\subsection{pltrun环境 - 显示\LaTeX{}代码}

有的时候,可能需要显示latex的代码和它运行的结果,这里提供了两个
一摸一样的环境,分别为\vb|pltrun|和\verb|pltRun|,就像这样:
\begin{pltRun}
\begin{pltrun}
\pltnte{测试}
\end{pltrun}
\end{pltRun}
或者
\begin{pltrun}
\begin{pltRun}
\pltrit{测试}
\end{pltRun}
\end{pltrun}

使用两个不一样的命令的原因是试图伪嵌套使用这个环境,而出于\LaTeX
的原因,它会在它匹配的第一个\vb|\end|命令出停止,从而报错。
所以两个一样的命令(除了大小写不一样之外)的目的就是为了嵌套使用。
实际使用的话其实用哪个的效果都是一摸一样的。

不过,伪嵌套也最多能够嵌套两层,如果想嵌套多层的话,还是应该在
拷贝下来的\vb|peterlitsdoc.cls|中更改来拷贝出更多的命令。(不过
应该没有人这么干吧)

%%%%%%%%%%%%%%%%%%%%%%%%%%%%%%%%%%%%%%%%%%%%%%%%%%%%%%%%%%%

\subsection{pltpic命令 - 显示图片}

使用]\vb|\pltpic|来显示图片,接受的参数是文件名,标题和引用名。
需要注意的是,浮动体内不能放在盒子里。

图\ref{head}的相应代码是:

\begin{lstlisting}
图\ref{head}的相应代码是:

\pltpic[0.4]{./temp.jpg}{头像}{head}
\end{lstlisting}

\pltpic[0.4]{./temp.jpg}{头像}{head}

其中命令\vb|\pltpic[width]{path}{title}{refname}|有一个可选参数,
默认为0.85,表示关于整个文本区的比例宽度。\vb|path|则是它的文件路径。
\vb|title|是它的小标题,而\vb|refname|是引用名字,可以被\vb|\ref|
命令调用得到对应的序号。

%%%%%%%%%%%%%%%%%%%%%%%%%%%%%%%%%%%%%%%%%%%%%%%%%%%%%%%%%%%

\subsection{plttodo命令 - todo格子}

使用\vb|plttodo|命令会显示一个to-do框框。应该会比较棒吧。
格式应该是\vb|\plttodo[<char>]|,如果\vb|char|是\vb|v|的话,
就是已经完成的框框,如果没有完成的话,就应该把框框搞成空格。
如果处于叠加态的话,就应该把\vb|char|设置为\vb|x|,这个时候,
它是一个半完成没有特别完成的状态中。如果是其他情况下的话,
那它就是一种不合法的状态。显示为有问号的框框。

因为可能会经常用到,它有一个别名,是\vb|\pltt|。

\begin{pltrun}
使用\plttodo[v]来标注已经完成的对象!
使用\plttodo[x]来标注快完成的对象。

\plttodo[v]喂猫咪

\plttodo[x]做数学作业

\plttodo[ ]跑步

它会比较第一个字符,然后根据第一个字符
来显示不同的内容。

\pltt[]   \pltt[x]  \pltt[  ] \pltt[ ]
\pltt[aa] \pltt[ a] \pltt[v]  \pltt[vv]
\pltt[ v]
\end{pltrun}


%%%%%%%%%%%%%%%%%%%%%%%%%%%%%%%%%%%%%%%%%%%%%%%%%%%%%%%%%%%

\subsection{pltbox环境 - 表格,或者选项}

定义了\vb|pltbox|环境,可以用来更好的定义制表符环境。
因为是在\vb|tabbing|的外面新添加了一个命令\vb|\col|所以说
基本上一样的。可以自己谷歌一下\vb|tabbing|。

\begin{pltrun}
\begin{pltbox}
\col{3}{1}\=\col{3}{1}\=\col{3}{1}\kill
A. this   \>B. that   \>C. help   \\
\plttodo[ ]apple                  \>
\plttodo[x]water                  \>
\plttodo[ ]kiss                   \\
\end{pltbox}
\end{pltrun}


%%%%%%%%%%%%%%%%%%%%%%%%%%%%%%%%%%%%%%%%%%%%%%%%%%%%%%%%%%%

\subsection{颜色}

提供了一系列简单的颜色:

\begin{pltrun}
\pltred   红色\pltrule
\pltblack 黑色\pltrule
\pltblue  蓝色\pltrule
\pltgray  灰色\pltrule
\end{pltrun}

%%%%%%%%%%%%%%%%%%%%%%%%%%%%%%%%%%%%%%%%%%%%%%%%%%%%%%%%%%%
%%%%%%%%%%%%%%%%%%%%%%%%%%%%%%%%%%%%%%%%%%%%%%%%%%%%%%%%%%%

\section{peterlitsdoc的更改}

所有的更改都是基于文档类\vb|ctexart|之上。

%%%%%%%%%%%%%%%%%%%%%%%%%%%%%%%%%%%%%%%%%%%%%%%%%%%%%%%%%%%

\subsection{边距}

把原来的窄边距更改的稍微大了一些,默认值为\vb|10pt|和
\vb|a4paper|。

%%%%%%%%%%%%%%%%%%%%%%%%%%%%%%%%%%%%%%%%%%%%%%%%%%%%%%%%%%%

\subsection{代码摘录}

原来的代码摘录环境不打算换行,尤其是行内的代码环境。为了
更棒的排版选择,我认为让它能够换行更加合理一些。

我使用了宏包\vb|lstlisting|来代替默认的\verb|\verb|命令,
现在使用\verb|\verb|命令的话,底层其实是\vb|lst-inline|,
会显得更加美观一些。

%%%%%%%%%%%%%%%%%%%%%%%%%%%%%%%%%%%%%%%%%%%%%%%%%%%%%%%%%%%s

\subsection{列表}

设置了列表环境\vb|enumerate|的一些长度。

%%%%%%%%%%%%%%%%%%%%%%%%%%%%%%%%%%%%%%%%%%%%%%%%%%%%%%%%%%%

\subsection{代码环境}

和\verb|\verb|一样,环境\vb|lstlisting|也是属于宏包
\vb|listings|,所以说代码环境推荐用\vb|lstlisting|来表
示代码:

\begin{pltrun}
如果代码过长的话会换行哦。
\begin{lstlisting}
from math import pi

fn main() {
    using namespace std;
    cout << "this is a code" << endl;

    0 # return code 0 meaning ok
}
\end{lstlisting}

行距,字号都会有一点点变化。
\end{pltrun}

字号可以刚刚好在正文中包含80个字符。
\begin{lstlisting}
--------------------------------------------------------------[ 80 characters ]
\end{lstlisting}

%%%%%%%%%%%%%%%%%%%%%%%%%%%%%%%%%%%%%%%%%%%%%%%%%%%%%%%%%%%

\subsection{默认宏包}

提供了一些常用的宏包,这样或许能够摆脱长长的\vb|\usepackage|
命令。

\begin{enumerate}
    \item calc          \hfill 在命令中使用数学表达式
    \item xcolor        \hfill 使用颜色
    \item mdwlist
          \hfill 更好看的列表,支持命令\vb|\pltpara|
    \item verbatim      \hfill 摘录环境,支持命令\vb|\pltrun|
    \item listings      \hfill 摘录环境,重定义命令\verb|\verb|
    \item enumitem      \hfill 列表环境
    \item hyperref      \hfill 更好的引用
    \item tikz          \hfill 流行的画图包
    \item graphicx      \hfill 可以简单的用图片了
    \item titlesec      \hfill 自定义的section样式
    \item url           \hfill 输入可以打开的url
\end{enumerate}

%%%%%%%%%%%%%%%%%%%%%%%%%%%%%%%%%%%%%%%%%%%%%%%%%%%%%%%%%%%

\subsection{编译环境}

作为一个面向中文环境的包,规定了使用XeLaTeX来作为默认的编
译环境,如果不使用,可能会报错哦。

%%%%%%%%%%%%%%%%%%%%%%%%%%%%%%%%%%%%%%%%%%%%%%%%%%%%%%%%%%%

\pltendant{2020}{5}{26}{Peterlits Zo}

\end{document}

